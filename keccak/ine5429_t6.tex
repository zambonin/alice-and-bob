\documentclass{article}

\usepackage[brazil]{babel}
\usepackage[T1]{fontenc}
\usepackage[a4paper, left=1.5cm, right=1.5cm, top=2cm, bottom=2cm]{geometry}
\usepackage[colorlinks, urlcolor=blue, citecolor=red]{hyperref}
\usepackage[utf8]{inputenc}
\usepackage[pdftex]{graphicx}
\usepackage{amsfonts, amsmath, enumitem, fancyhdr, float, subcaption}

\pagestyle{fancy}

\fancyhead{}
\fancyfoot[L]{\href{https://github.com/zambonin/ufsc-ine5429}{\texttt{src/}}}
\fancyfoot[C]{\itshape\nouppercase{\leftmark}}
\fancyfoot[R]{\thepage}

\renewcommand{\headrulewidth}{0pt}
\renewcommand{\footrulewidth}{0.75pt}

\fancypagestyle{firststyle} {
    \renewcommand{\headrulewidth}{0.75pt}
    \fancyhead[L]{UFSC -- INE5429}
    \fancyhead[C]{Gustavo Zambonin}
    \fancyhead[R]{\footnotesize{\texttt{gustavo.zambonin@grad.ufsc.br}}}
}

\title{\vspace{-1.5cm}
    \Large{\textbf{
        A função \emph{hash} criptográfica SHA-3
    }}\vspace{-1.5cm}
}
\author{}
\date{}

\begin{document}

\maketitle

\thispagestyle{firststyle}

\section{Definições}

\begin{itemize}

\item Uma \emph{função hash criptográfica}, ou função de resumo criptográfica
(futuramente denotada por $h$), é um algoritmo matemático que mapeia uma
quantidade de bytes qualquer\footnote{algumas funções desse tipo têm
limites quanto ao tamanho da entrada, embora estes sejam extremamente
grandes.} para uma palavra de tamanho fixo, ou seja,
$h : \{0, 1\}^{*} \longrightarrow \{0, 1\}^{n}$, $n \in \mathbb{N}$.

Para que seja resistente a diversos tipos de criptoanálise, uma função
$h : X \longrightarrow Y$ deve respeitar algumas propriedades:

\begin{enumerate}[label=\roman*.]

\item \emph{Resistência à pré-imagem}: Para um resumo $M' \in Y$, é
computacionalmente impraticável\footnote{o tempo ou recursos gastos para esta
computação excedem a validade ou utilidade da informação desejada.} encontrar
a mensagem $M \in X$ tal que $h(M) = M'$. Uma função matemática com esta
propriedade é chamada de unidirecional.

\item \emph{Resistência à segunda pré-imagem}: Para uma mensagem $M_0 \in X$,
é computacionalmente impraticável encontrar uma segunda mensagem $M_1 \in X$
tal que $M_0 \neq M_1$ e $h(M_0) = h(M_1)$.

\item \emph{Resistência à colisão}: Para duas mensagens $M_0, M_1 \in X$, é
computacionalmente impraticável encontrar $M_0 \neq M_1$ e $h(M_0) = h(M_1)$.

\end{enumerate}

É importante notar que, embora as definições sejam extremamente parecidas,
resistência à segunda pré-imagem e resistência à colisão são conceitos
diferentes; um atacante não consegue escolher a primeira mensagem caso queira
atacar a resistência à segunda pré-imagem; para a resistência à colisão, o
atacante pode escolher livremente o par de mensagens.

\item Algumas aplicações destas funções são enumeradas abaixo:

\begin{itemize}

\item Podem ser utilizadas para verificar a integridade da mensagem, comparando
resumos criptográficos calculados antes e depois da transmissão de mensagem
e/ou arquivos.

\item Para evitar o armazenamento de senhas em texto claro, é possível
armazenar apenas o resumo criptográfico de cada senha e compará-lo na
autenticação do usuário.

\item Resumos criptográficos são comumente descritos como identificadores
únicos seguros para um arquivo ou informação digital (por exemplo,
\emph{commits} em um sistema de controle de versão).

\end{itemize}

\item O padrão SHA-3, descrito pelo documento FIPS 202 \cite{Dworkin2015}, é
baseado em uma instância da família \textsc{Keccak} de permutações matemáticas,
selecionada pelo NIST (\emph{National Institute of Standards and Technology})
e especificada neste documento.

\end{itemize}

\section{O algoritmo SHA-3}

\begin{itemize}

\item \textsc{Keccak} é uma família de funções esponja. Este tipo de função é
uma generalização do conceito da função de resumo criptográfica com saída
infinita. Após a aplicação de uma função de preenchimento (\emph{padding}) à
mensagem $M$, a função esponja tem duas fases: a fase de absorção
(\emph{absorbing}), responsável por intercalar blocos de $M$ com aplicações de
uma função de permutação $f$, de modo iterativo; e a fase de compressão
(\emph{squeezing}), onde os blocos de saída, intercalados novamente pela
permutação $f$, são concatenados para gerar uma palavra com um número de bits
configurável pelo usuário. Esse processo pode ser observado na figura
\ref{fig:sponge}.

\item A permutação $f$ é descrita como uma sequência de operações num estado
$A$, que é um vetor de elementos tridimensional em $GF(2)$.
$f$ é uma permutação iterativa, consistindo de uma sequência de rodadas.
Uma rodada $R$ consiste da composição de cinco etapas:
$R = \iota \circ \chi \circ \pi \circ \rho \circ \theta$, como visto
em \ref{fig:steps}:

\begin{enumerate}[label=\roman*.]

\item A etapa $\theta$ faz a soma \texttt{XOR} de um elemento de $A$ e todos
os elementos das colunas adjacentes indicadas.

\item A etapa $\rho$ dispersa os elementos entre cortes transversais verticais
de $A$.

\item A etapa $\pi$ rearranja as posições de elementos em cortes transversais
horizontais de $A$.

\item A etapa $\chi$ tem como efeito fazer a soma \texttt{XOR} de cada bit em
uma linha, de acordo com uma função não-linear de dois outros bits adjacentes.

\item A etapa $\iota$ é utilizada para quebrar a simetria das operações acima,
e sem esta etapa, todas as rodadas teriam a mesma saída. A soma \texttt{XOR} de
alguns bits do estado $A$ é feita com um bit específico de uma sequência gerada
por um LFSR\footnote{\emph{linear-feedback shift register}, um tipo de gerador
de sequências pseudoaleatórias.}, alimentado pelo índice da rodada atual.

\end{enumerate}

\begin{figure}[H]
    \centering
    \includegraphics[scale=0.75]{sponge}
    \caption{Uma construção esponja. Imagem retirada de \cite{SpongeReference}.}
    \label{fig:sponge}
\end{figure}

\begin{figure}[H]
    \begin{subfigure}{.5\textwidth}
        \centering
        \includegraphics[scale=0.8]{theta_step}
    \end{subfigure}%
    \begin{subfigure}{.5\textwidth}
        \centering
        \includegraphics[scale=0.35]{pi_step}
    \end{subfigure}
    \begin{subfigure}{.5\textwidth}
        \centering
        \includegraphics[scale=0.7]{rho_step}
    \end{subfigure}%
    \begin{subfigure}{.5\textwidth}
        \centering
        \includegraphics[scale=0.65]{chi_step}
    \end{subfigure}
    \caption{Da esquerda para a direita e de cima para baixo, as etapas
    $\theta$, $\pi$, $\rho$ e $\chi$. Imagens retiradas de
    \cite{KeccakReference}.}
    \label{fig:steps}
\end{figure}

\end{itemize}

\bibliography{ine5429_t6}
\bibliographystyle{plain}

\end{document}
